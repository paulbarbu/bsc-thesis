\documentclass[12pt,a4paper,titlepage]{report}

\usepackage{times}

\usepackage[utf8]{inputenc}
\usepackage[T1]{fontenc}
\usepackage[romanian]{babel}

\usepackage{setspace}
\onehalfspacing

\usepackage[hidelinks]{hyperref}

\author{Barbu Paul - Gheorghe}

%TODO: coperta
%TODO: plan tematic?

\begin{document}

\begin{titlepage}
{
\centering
UNIVERSITATEA "LUCIAN BLAGA" DIN SIBIU\\
FACULTATEA DE INGINERIE \\
DEPARTAMENTUL DE CALCULATOARE ŞI INGINERIE ELECTRICĂ\\
}

\vfill

{
\centering
\Large
Secured Bootloader: program pentru controlul secvenței de
inițializare și al actualizării aplicației în microcontrollere
programabile
}

\vfill
{
\raggedright
Conducător ştiinţific: TODO: c.d. cu  grad şi titlul de doctor \\
Îndrumător: TODO: c.d. ce s-a ocupat de student
}
\vfill

{
\raggedright
\hspace*{160pt}Absolvent: \\
\hspace*{160pt}Barbu Paul - Gheorghe\\
\hspace*{160pt}Specializarea:\\
\hspace*{160pt}Calculatoare și Tehnologia Informației
}

\end{titlepage}

\tableofcontents
\newpage

\subsection*{Prezentarea temei}

Lucrarea de față își propune dezvoltarea unui bootloader securizat pentru dispozitive embedded, aceste dispozitive conțin un microcontroller și în general utilizatorul respectivelor dispozitive nu vede și nu interacționează direct cu microcontroller-ul. Exemple de dispozitive embedded sau de aparate care conțin asemenea dispozitive: frigider, mașină de spălat, autoturism, termostat, contor electric, sistem de alarmă, sistem de supraveghere, ceasuri, etc.

Un bootloader este o aplicație care este în general încărcată pe microcontroller de către firma care vinde microcontroller-ul și care nu poate fi înlocuită ușor de către utilizatorii acestuia.
Rolul acestei aplicații este de a fi gazda ("host"-ul) și de a face verificările necesare încărcării unei alte aplicații pe microcontroller. În urma verificărilor se poate afirma că aplicația "guest" sau utilizator este una validă și că poate fi rulată de microcontroller cu încredere.
În urma procesului de bootloading ambele aplicații, atât bootloader-ul cât și aplicația utilizator, vor coexista în
memoria non-volatilă a dispozitivului embedded\footnote{în general pe flash-ul microcontroller-ului, conform unui memory map predefinit de producător}.
 
Motivele pentru care bootloader-ul este securizat și pentru care înlocuirea lui sau a aplicației utilizator
nu ar trebui să poată fi făcută de oricine țin în mică parte de protejarea proprietății intelectuale a celor ce creează aplicații guest pentru respectivul dispozitiv și într-o mai mare măsură țin de asigurarea integrității și a evitării modificării neautorizate a acestei aplicații pentru a asigura utilizatorului experiența pe care producătorul dispozitivelor embedded și-a imaginat-o. Acest lucru este crucial în ziua de astăzi, când criminalitatea în mediul cibernetic este în creștere\cite{interpol}\cite{morgan2016}. Consumatorii trebuie să poată fi protejați proactiv de acțiunile ilegale ale crackerilor, în antiteză cu metoda tradițională, reactivă care constă în schimbarea și îmbunătățirea unui sistem după ce acesta a fost compromis. Cele două metode se complementează reciproc, dar metoda proactivă are avantajul prevenirii unei acțiuni malițioase asupra sistemului informatic.

În afară de îmbunătățirea securității unui sistem embedded precum a fost descris mai sus, rolul bootloader-ului ar trebui să fie și de a mări flexibilitatea unui astfel de sistem. Acest rol este îndeplinit prin faptul că aplicația utilizator de pe microcontroller poate fi schimbată fără modificări hardware asupra sistemului. Schimbarea aplicației utilizator poate surveni din diverse motive, printre care: dezvoltatorul sistemului embedded hotărăște că este timpul să introducă funcționalități noi în aplicație sau că aplicația veche are unele defecte (bug-uri) care trebuie reparate sau chiar că sistemul embedded care înainte îndeplinea un rol trebuie să îndeplinească un rol complet nou, diferit față de cel vechi. Toate aceste scenarii implică folosirea bootloader-ului pre-instalat pe un dispozitiv pentru a încărca o altă aplicație (fie că e una complet nouă sau doar puțin modificată datorită remedierii defectelor din versiunea anterioară).

Bootloader-ul dezvoltat ar trebui să ne permită de asemenea reducerea costurilor. Prin costuri putem să considerăm de exemplu costurile survenite în urma livrării unui sistem embedded. Prin definiție mecanismele interne ale unui astfel de sistem nu sunt accesibile consumatorilor, deci orice modificare, fie ea chiar și software trebuie să fie făcută de producător. Așadar dispozitivul trebuie să fie livrat la producător, acesta aplică modificările și livrează înapoi dispozitivul clientului. Prin intermediul bootloader-ului ne propunem să putem face aceste modificări software la distanță. În esență aceste modificări se vor putea face independent de către fiecare utilizator al sistemului.

Succint cerințele pe care trebuie să le satisfacă bootloader-ul sunt:
\begin{itemize}
\item asigurarea flexibilității dispozitivului embedded prin faptul ca aplicațiile de utilizator se pot schimba la nevoie
\item asigurarea securității și integrității unui sistem embedded în momentul în care aplicația este încărcată\footnote{securitatea sistemului depinde 100\% de aplicație în momentul în care aceasta rulează}
\item minimizarea costurilor de întreținere a dispozitivelor embedded dintr-un mediu industrial prin posibilitatea de a acționa bootloader-ul din exteriorul sistemului embedded
\end{itemize}

Prin rezolvarea acestor puncte, bootloader-ul va fi de ajuns de generic pentru aplicabilitate în diverse domenii și pe diverse platforme hardware. Fiind vorba de mediul embedded, bootloader-ul a fost proiectat și dezvoltat cu constrângerile de mărime în minte, cu atât mai mult cu cât pe un microcontroller se vor stoca două aplicații simultan, bootloader-ul și aplicația utilizator, dar doar una va rula la un moment dat.

\newpage
\chapter{Considerații teoretice}
\section{Introducere și etimologie}
Un bootloader se află pe orice sistem de calcul care trebuie să încarce un sistem de operare sau o aplicație terță.
Acest lucru este valabil fie că utilizatorul interacționează cu sistemul în timpul procesului de "boot" fie că procesul se petrece automat, fără interacțiune umană.

Cuvântul bootloader vine de la procesul de "boot" (sau "boot up"), care la rândul său este o scriere prescurtată a "bootstrap". Bootstrap în engleză semnifică "to pull oneself up by one's bootstraps", procesul prin care noi pornim la drum încălțându-ne. \cite{wikiBootstrap}

\section{Modul de funcționare al unui bootloader}
Orice bootloader trebuie stocat într-o memorie non-volatilă și de preferat disponibilă doar pentru citire, o memorie ROM\footnote{Read-Only Memory}, deoarece acesta trebuie să fie disponibil în momentul în care sistemul pornește indiferent de ce s-a întâmplat anterior pe sistem. Acest lucru garantează funcționarea corectă a sistemului și faptul că sistemul de operare sau aplicația utilizator poate fi schimbată. În caz contrar, când sistemul încetează să funcționeze din cauza unui defect (în general software) putem spune ca este "brick-uit", de la englezescul "brick" (cărămidă)\cite{wikiBrick}.

Un bootloader este, fără excepție, primul program executat de către un calculator.
Codul este încărcat din ROM în RAM\footnote{Random - Access Memory}, este executat, iar în urma execuției acestui cod (în mod tipic un sector) se pot întâmpla unul din două lucruri:
\begin{enumerate}
\item bootloader-ul și-a încheiat treaba și a predat controlul aplicației utilizator respectiv sistemului de operare
\item bootloader-ul nu încape într-un sector și codul executat anterior este doar auxiliar
\end{enumerate}
În al doilea caz avem de a face cu un așa zis "two stage bootloader", în primul stagiu făcându-se doar apelul spre acest stagiu secundar. Aici se va face de fapt încărcarea aplicației utilizator. Motivul din spatele încărcării în două faze este faptul că, fără drivere de disc și fără sisteme de fișiere, unitatea maximă pe care procesorul o poate citi de pe hard disc este sectorul.

Posibila diferență la sistemele embedded este aceea că microcontroller-ul poate executa cod direct din memoria flash și că bootloader-ul este probabil mai mic ca un sector, deoarece aplicațiile sunt reduse ca dimensiune și ca periferice cu care pot interacționa, display-ul și tastatura nefiind necesare în astfel de sisteme în timpul procesului de boot up.

La calculatoarele personale un bootloader bine cunoscut și pentru care avem nevoie de display și tastatură este GNU GRUB, folosit de majoritatea distribuțiilor GNU/Linux. Acest bootloader este vizibil utilizatorului și se poate chiar schimba tipul de sistem de operare încărcat (folosind tastatura).

TODO: GRUB image
TODO: %https://msdn.microsoft.com/en-us/library/ms902397.aspx
TODO: cum se intampla boot-ul la PC + POST % https://en.wikipedia.org/wiki/Booting\#Boot_sequence
TODO:% https://en.wikipedia.org/wiki/Master_boot_record\#Programming_considerations
TODO: despre alte bootloadere la PC: grub, lilo

TODO: despre cum se scrie codul in lumea embedded, fara bootloader prima data (flashuirea din fabrica)
TODO: despre alte bootloadere embedded

\newpage
\chapter{Rezolvarea temei de proiect}
\section{Introducere}
TODO: memory map, cum sta aplicatia host si guest in memorie
TODO: scrie-ma

\section{Concluzii și dezvoltări ulterioare}
TODO: scrie-ma


\newpage
\bibliographystyle{plain}
\bibliography{bibliografie}

\end{document}
